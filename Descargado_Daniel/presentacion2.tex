%\vspace*{0.1cm}
%\begin{center}
%{\LARGE\bf  \sffamily\textcolor{ocre}{Presentación}}
%\end{center}

%\vskip 0.6cm
%\chapterimage{/home/camilo/TEMP/scapp/bulletfiles/sgc_blanco2.jpg} % Chapter heading image

%\pagestyle{plain}
\chapter{PRESENTACIÓN}
%\vspace*{4 cm}
\noindent La Red Sismológica Nacional de Colombia - RSNC, adscrita al SERVICIO GEOLÓGICO COLOMBIANO, es la encargada de observar, monitorear, investigar y evaluar la actividad sísmica del país de manera continua y permanente, con el fin de conocer el fenómeno sísmico y la amenaza que éste representa. Así mismo, ante la ocurrencia de un evento destacado, brindar información oportuna al Sistema Nacional de  Gestión del Riesgo de Desastres (SNGRD), a las diferentes entidades y a la comunidad en general.\\\\

%\vspace*{0.5cm}

\noindent El propósito del boletín mensual es presentar un resumen de los parámetros y localizaciones preliminares de los sismos registrados por la RSNC en el territorio colombiano durante el periodo respectivo. Información general de la Red Sismológica, así como la sismicidad general registrada desde junio de 1993, se puede consultar en nuestra página web http://www.sgc.gov.co. \\\\


%\vspace*{0.5cm}

\noindent Cordialmente,  \\\\

%\vspace*{0.4cm}

\noindent M.Sc. Viviana Dionicio Lozano.\\ 
{\bf \sffamily\textcolor{ocre}{Coordinadora.}}\\ 
Grupo de evaluación y monitoreo de la actividad sísmica.\\

%\vskip 0.8cm
\pagebreak

%\input{anexos/autores}
\begin{center}
\vspace*{2cm}
{\bf \sffamily\textcolor{ocre}{Director General del SERVICIO GEOLÓGICO COLOMBIANO}}\\
Dr. Oscar Paredes Zapata\\ 
\vspace{0.5cm} 
{\bf \sffamily\textcolor{ocre}{Directora técnica de Geoamenazas}} \\
Dra. Marta Lucia Calvache\\ 
\vspace{0.5cm}
%\begin{center} 

{\bf \sffamily\textcolor{ocre}{RED SISMOLÓGICA NACIONAL DE COLOMBIA}}\\
%\end{center}
\vspace{0.4cm}
\end{center}

\begin{multicols}{2}
\begin{flushleft}

{\bf \sffamily\textcolor{ocre}{Coordinadora  de la Investigación}} \\ 
{\bf \sffamily\textcolor{ocre}{y el Monitoreo de la Actividad Sísmica}} \\ 
M.Sc. Viviana Dionicio Lozano\\
\vspace{0.4cm}


{\bf \sffamily\textcolor{ocre}{ANALISTAS}}\\
Geól. Miguel Ángel Cubillos. \\
Ing. Oscar Daniel Suárez Mejía. \\
Geól. Laura Vanessa Velásquez., Est. M.Sc. \\
Ing. Leonardo Santos Mateus Báez., Est. M.Sc. \\
Físico. Daniel David Siervo Plata., Est. M.Sc. \\
Geól. Daniel Martínez Jaramillo., Est. M.Sc. \\
Ing. Jhon Leandro Pérez., M.Sc. \\
Geól. Daniela Hernández Beltrán., Est. M.Sc. \\
\vspace{0.4cm}

{\bf \sffamily\textcolor{ocre}{ÁREA DE SISMOLOGÍA}}\\ 
Física. Patricia Pedraza., M.Sc. \\
Físico. Camilo Muñoz Lopez. \\
Ing. Fís. Ruth Emilse Bolaños.\\
Físico. Juan Santiago Velásquez., M.Sc. \\
Geológa. Lina Paola Aguirre., M.Sc. \\
Lic. Física. Edwin Fabián Mayorga López., M.Sc.\\
Lic. Física. Hugo Esteban Poveda., M.Sc.\\
Geológo. Omar Alfredo Mercado Días., M.Sc.\\
\vspace{0.4cm}

{\bf \sffamily\textcolor{ocre}{ÁREA DE ELECTRÓNICA}}\\
Ing. Andres Felipe Gómez\\
Ing. Jorge Andres de la Rosa.\\
Ing. Juan Carlos Lizcano.\\
Ing. Johnnatan Fernandez.\\ 
Tec. Edgar Gil.\\
Tec. Robert Prada.\\
Ing. Ariel Portocarrero.\\
\vspace{0.4cm}

{\bf \sffamily\textcolor{ocre}{ÁREA DE SISTEMAS}}\\  
Ing. Sist. Monica Yaneth Acosta., Esp.\\
Ing. Sist. Orlando Chamorro. \\
Ing. Sist. Oscar David Riobamba. \\
Ing. Sist. Carlos Guillermo Araujo Moncayo \\
\end{flushleft}
\end{multicols}
