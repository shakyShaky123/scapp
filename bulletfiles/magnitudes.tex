\begin{exercise}
\sffamily
El promedio ponderado de las magnitudes se calcula de la siguiente manera:\\
\begin{equation*}
M = \frac{\sum w_iM_i}{\sum w_i}
\end{equation*}

Donde $w_i = a_i conteo\_sta(M_i)+b_i $. Los tipos de magnitud utilizados son:
\begin{itemize}
\item \textbf{MLr}: Magnitud local adecuada para colombia (F. Rengifo y A. Ojeda, 2004).
\item \textbf{Mw(mB)}: Estimación de magnitud Mw en base a la magnitud mB a través 
de la regresión Mw vs mB (Bormann y Saul, 2008).
\item \textbf{Mw(Mwp)}: Estimación de magnitud Mw en base a a la magnitud Mwp a través
de la regresión Mw vs Mwp (Whitmore, 2002).
\end{itemize}

\noindent
Las magnitudes Mw son consideradas para el cálculo de la magnitud promedio si se encuentran
en almenos 4 estaciones. En caso contrario se deja la magnitud local.
Para cualquier duda o comentario sobre el cálculo de magnitudes puede escribirnos a
\textit{sismologo@sgc.gov.co}.
\end{exercise}%
